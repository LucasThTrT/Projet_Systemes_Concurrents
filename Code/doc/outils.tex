%===============================================================================
%
%===============================================================================
\section{Les principaux outils}

   Le but de cette section est de décrire brièvement les outils
utilisés pour compiler et exécuter \manux.

   L'essentiel du code de \manux est écrit en C, et un peu
d'assembleur est nécessaire ici ou là, \ldots Le processus de
compilation est évidemment automatisé avec l'outil {\em make}. Je vais
décire sommairement cette étape dans la sous-section
\ref{subsection:compilation}.

   \manux peut ensuite être exécuté sur une machine réelle, mais il
est évidemment possible d'utiliser un émulateur. Nous évoquerons les
possibilités dans la sous-section \ref{subsection:execution}.

   Le débugage est un élément important dans la phase de développement
d'un logiciel, et il se révèle parfois un peu délicat lorsqu'il s'agit
de développer un noyau. Nous évoquerons des outils dans les
sous-sections \ref{subsection:debugage_interne} et
\ref{subsection:debugage_externe}.

%...............................................................................
%
%...............................................................................
\subsubsection{Le compilateur C}

   Le compilateur GCC est utilisé, la version 11.3.0 avec une cible
\lstinline!x86_64-linux-gnu! fonctionne parfaitement. Je n'ai jamais
eu de problème avec aucune version de GCC jusqu'à présent.

%...............................................................................
%
%...............................................................................
\subsubsection{L'assembleur}

   J'utilise \nasm \autocite{nasm-website}  pour les quelques fichiers
écrits en assembleur, dans sa version 2.14.
 
%-------------------------------------------------------------------------------
%
%-------------------------------------------------------------------------------
\subsection{La compilation}

   La commpilation générale du noyau est fondée sur l'outil {\tt
make}, et j'utilise la version 4.2.1 de \gnu {\tt make}.

%-------------------------------------------------------------------------------
%
%-------------------------------------------------------------------------------
\subsection{Les scripts}

   Quelques outils de scripts tels que {\tt awk} sont utilisés, rien
de très original, et tout système Linux correctement installé devrait
fournir ce qui est nécessaire.

%-------------------------------------------------------------------------------
%
%-------------------------------------------------------------------------------
\subsection{La gestion des fichiers {\sc iso}}

   Les fichiers {\sc iso} son générés par la commande {\tt
     grub-mkrescue} qui utilise {\tt xorriso}.
   
%-------------------------------------------------------------------------------
%
%-------------------------------------------------------------------------------
\subsection{L'exécution dans un émulateur}

   Bien sûr, l'idée n'est pas de faire fonctionner ManuX sur du
matériel réel, même si c'est tout à fait possible, mais plutôt dans un
émulateur.

   J'utilise pour cela \qemu dans sa version 3.1.0. On peut
également utiliser {\tt bochs} ainsi que VirtualBox.
   
%-------------------------------------------------------------------------------
%
%-------------------------------------------------------------------------------
\subsection{La documentation}

   La maigre documentation (que vous êtes  en train de lide) est
écrite essentiellement en LaTeX. N'importe quelle version convient
parfaitement.

   J'utilise également DoxyGen et mkdocs pour générer ensuite une
version navigable en ligne.

%===============================================================================
%
%===============================================================================
\section{Compilation de ManuX}

   Le {\tt Makefile} principal défini un certain nombre de cibles
utilisables pour compiler et exécuter ManuX.
   
%-------------------------------------------------------------------------------
%
%-------------------------------------------------------------------------------
\subsection{La création du noyau}

   Sans grande surprise, on compile avec la commande

\begin{lstlisting}
  make
\end{lstlisting}

  Différentes cibles peuvent être créées

\begin{itemize}
   \item {\tt manux} permet de compiler une image du noyau ;
   \item {\tt run} permet de lancer un émulateur (par défaut pour le moment c'est \qemu) ;
   \item {\tt clean} permet d'effacer tous les fichiers issus de compilation ;
   \item {\tt iso} permet de créer une image \iso.
\end{itemize}

%-------------------------------------------------------------------------------
%
%-------------------------------------------------------------------------------
\subsection{Gestion de configurations multiples}

   L'essentiel de la configuration de ManuX se fait dans le fichier
\fichier{include/manux}{config}{h}. Il est cependant possible de
demander à \make d'utiliser un fichier différent.

   Le répertoire {\tt multiconf} contient différents fichiers de
configuration. Chacun de ces fichiers permet alors de générer une
version différente de ManuX.

   Ainsi, le fichier \fichier{multiconf}{printk}{h} définit la
configuration permettant de compiler un noyau ManuX montrant
l'utilsation de {\tt printk}. On le compile de la façon suivante

\begin{lstlisting}
  make CFG=printk
\end{lstlisting}

   Il peut alors être exécuté par exemple avec

\begin{lstlisting}
  make run
\end{lstlisting}

%-------------------------------------------------------------------------------
%
%-------------------------------------------------------------------------------
\subsection{Création d'un fichier \iso}

   Il est également possible de créer un fichier \iso contenant une ou
plusieurs configuration de ManuX.

   Pour créer un fichier \iso contenant la version actuellement
définie dans le fichier \fichier{include/manux}{config}{h} on
utilisera simplement

\begin{lstlisting}
  make iso
\end{lstlisting}

   L'exécution pourra alors se faire via la commande

\begin{lstlisting}
  make runiso
\end{lstlisting}

   Il est également possible de générer un fichier \iso contenant
toutes les versions de ManuX décrites par les fichiers du répertoire
\lstinline!multiconf! de la façon suivante

\begin{lstlisting}
  make multiso
\end{lstlisting}

On l'exécutera de la même façon :

\begin{lstlisting}
  make runiso
\end{lstlisting}
   
%-------------------------------------------------------------------------------
%
%-------------------------------------------------------------------------------
\subsection{L'édition de liens}

   Afin d'assembler tous les éléments en un seul fichier
\lstinline!noyau.elf! qui pourra ensuite être utilisé par {\em
  Multiboot}, nous allons utiliser l'éditeur de liens \lstinline!ld!.
Le comportement de \lstinline!ld! est conditionné par un fichier qui
s'appelle un {\em linker script}. Un tel script décrit en particulier
comment les différentes sections des fichiers d'entrée sont disposées
dans le fichier de sortie et comment la mémoire est organisée.

%https://wiki.osdev.org/Linker_Scripts

Le code suivant montre un exemple simple de script :

\begin{lstlisting}
ENTRY(start)
SECTIONS {
    . = 1M 
    .boot :
    {
        *(.multiboot)
    }
    adresseDebutManuX = .;
    .text : {
       *(.text)
    }        
    .rodata : {
       *(.rodata)
    }
    .data : {
       *(.data)
    }
    .bss : {
       *(.bss)
    }
}
\end{lstlisting}

   Dans ce script, nous avous défini le nom du point d'entrée (ici {\tt
  start}), et nous avons décrit un certain nombre de sections avec
leur ordre d'apparition dans le fichier de sortie et leur
contenu. Vous remarquerez également que l'on utilise la notation
``{\tt .}'' pour spécifier l'adresse courante et que cette adresse
peut être consultée mais également modifiée. Ainsi la ligne {\tt . =
  1M} stipule que l'on se place ) l'adresse 1M. D'un autre côté la
ligne {\tt adresseDebutManuX = .} stipule que l'on enregistre dans la
variable {\tt adresseDebutManuX} la valeur de l'adresse courante
(celle du début de la section {\tt . text} donc).


%===============================================================================
%
%===============================================================================
\section{Exécution de ManuX}
\label{section:execution}

   Le but n'est évidemment pas de faire tourner cette merveille sur du
vrai matériel, mais plutôt dans un émulateur.

%...............................................................................
%
%...............................................................................
\subsection{L'émulateur Qemu}

   C'est maintenant \qemu \cite{qemu-website}  que j'utilise, par exemple de la façon
suivante
   
\begin{lstlisting}
   qemu-system-i386 -drive format=raw,file=manux,index=0,if=floppy -m 64M
\end{lstlisting}

  C'est d'ailleurs la commande que lance

\begin{lstlisting}
   make runfloppy
\end{lstlisting}

   Une autre façon de démarrer \manux est la suivante

\begin{lstlisting}
   qemu-system-i386 -kernel noyau.elf -m 64M
\end{lstlisting}

  C'est ce que fait la commande

\begin{lstlisting}
   make run
\end{lstlisting}

   Mais on préférera probablement passer par une image ISO :

\begin{lstlisting}
   qemu-system-i386 -m 64M -cdrom manux.iso
\end{lstlisting}

  C'est ce que fait la commande

\begin{lstlisting}
   make runiso
\end{lstlisting}

   De nombreuses options sont disponibles, mais ce n'est pas le moment
d'en parler !

%...............................................................................
%
%...............................................................................
\subsection{L'émulateur Bochs}

   J'utilisais initialement cet outil, mais les fichiers de
configuration ne semblent plus convenir aux dernières versions de
Bochs {} \cite{bochs-website} il faudrait les mettre à jour !

%===============================================================================
%
%===============================================================================
\section{Le débugage de ManuX}

%-------------------------------------------------------------------------------
%
%-------------------------------------------------------------------------------
\subsection{Le débugage ``interne''}
\label{subsection:debugage_interne}

   Ce que j'appelle ici le débugage interne, c'est tous les outils que
l'on peut utiliser dans le code lui-même pour essayer de comprendre ce
qu'il s'y passe.

   Nous avons pour le moment quelques outils de base, copieusement bugués
eux-mêmes (!).

%...............................................................................
%
%...............................................................................
\subsubsection{La fonction {\tt printk\_debug}}

   Elle permet de produire des messages d'erreur qui seront envoyés de
façon sélective en fonction de la valeur de \lstinline!masqueDebugage!
qui doit être construite dans {\tt manux/debug.h}.

%...............................................................................
%
%...............................................................................
\subsubsection{La macro {\tt assert}}

   Elle est définie dans le fichier \fichier{include/manux}{debug}{h}
et elle permet de vérifier {\em lors de l'exécution} qu'une propostion
est vérifiée ou non. Si elle ne l'est pas, un message est affiché.

%...............................................................................
%
%...............................................................................
\subsubsection{Le handler {\tt handlerPanique}}

   La fonction {\tt handlerPanique} est définie dans le fichier
\fichier{include/manux}{debug}{h} et permet d'afficher un état du
système qui peut être utile à des fin de débugage.

%...............................................................................
%
%...............................................................................
\subsubsection{Le fichier {\tt noyau/symbols}}

   Il va surtout nous servir pour aider au débugage externe, mais il
est généré à la compilation du noyau.
 
%-------------------------------------------------------------------------------
%
%-------------------------------------------------------------------------------
\subsection{Le débugage ``externe''}
\label{subsection:debugage_externe}

   Une technique pour débuguer est d'utiliser {\tt qemu} couplé avec
{\tt gdb}. Pour cela, je lance par exemple {\tt qemu} de la façon suivante

\begin{lstlisting}
   qemu-system-i386 -drive format=raw,file=manux,index=0,if=floppy  -rtc
base=localtime -m 64M -gdb tcp::1234  -S 
\end{lstlisting}

   Je lance ensuite {\tt gdb} dans un autre terminal avec un
\lstinline!.gdbinit! qui contient~:

\begin{verbatim}
  target remote :1234
  dir noyau
  dir lib
  file noyau/noyau.elf
\end{verbatim}

   La commande \lstinline!dir! permet de dire à {\tt gdb} où trouver
les fichiers sources et la command \lstinline:file: lui permet de
consulter le binaire du noyau pour y trouver l'adresse des fonctions
et variables (ici en fait, c'est une autre compilation du noyau au
format {\sc elf} !).

%...............................................................................
%
%...............................................................................
\subsubsection{Exécution}

   Quelques commandes de base pour l'exécution de code dans {\tt gdb}

\begin{itemize}
     \item {\tt {\bf r} un} permet de lancer l'exécution d'un
       programme (inutilisable ici étant donné notre fonctionnement :
       on débogue à distance) ;
     \item {\tt {\bf c} ontinue} permet de poursuivre l'exécution jusqu'au
       prochain point d'arrêt/bug/fin/\ldots ;
     \item {\tt {\bf s} tep} avance d'une instruction dans le code ;
     \item {\tt {\bf n} ext}
     \item {\tt {\bf b} reakpoint fonction}
     \item {\tt {\bf b} reakpoint fichier:ligne}
\end{itemize}

%...............................................................................
%
%...............................................................................
\subsubsection{Affichage des registres}

   Il peut être intéressant de consulter le contenu des registres

\begin{itemize}
     \item {\tt info registers} donne le contenu des principaux registres.
\end{itemize}

%...............................................................................
%
%...............................................................................
\subsubsection{Affichage des variables}

   {\tt Gdb} permet également de consulter le contenu des variables :

\begin{itemize}
   \item {\tt p toto} affiche le contenu de la variable {\tt toto}.
\end{itemize}

%...............................................................................
%
%...............................................................................
\subsubsection{Désassemblage du code}

    Une fonctionnalité intéressante est de pouvoir observer le code
assembleur, ce qui peut permettre de comprendre certains comportements
étranges, tels que les conséquences de choix d'optimisation du
compilateur :

\begin{itemize}
   \item {\tt disassemble fonc} affiche le code assembleur de la
     fonction {\tt fonc}.
\end{itemize}

